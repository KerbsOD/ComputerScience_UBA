% CONFIGURACIÓN DEL PROYECTO
\documentclass[a4paper]{article}
\setlength{\parskip}{2mm}
\newcommand{\tab}{~ \qquad}
\input{Algo1Macros}
\usepackage{caratula} 
\usepackage{hyperref}
\usepackage{scrextend}
\usepackage{blindtext}




% CARÁTULA
\begin{document}

\titulo{TP de Especificación}
\subtitulo{Trabajo Práctico Grupal}
\fecha{30 de Marzo de 2022}
\materia{Algoritmos y Estructuras de Datos I}
\grupo{Grupo 10}

\newcommand{\dato}{\textit{Dato}}
\newcommand{\individuo}{\textit{Individuo}}

\integrante{Dominguez, Emilia}{37752993}{maemiliadominguez@gmail.com}
\integrante{Kerbs, Octavio}{64/22}{octaviokerbs@gmail.com}
\integrante{Russo, Gabriel}{107/19}{gabrielrussoguiot@gmail.com}
\integrante{Traverso, Lucas}{479/18}{lucas6246@gmail.com}

\maketitle


% ÍNDICE
\tableofcontents
\newpage


% DEFINICIÓN DE TIPOS
\section{Definición de Tipos}
\begin{description}
	\item type \textit{pos} = \(\ent \times \ent\) 
	\item type \textit{tablero} = \( \TLista{\TLista{\bool}}\) 
	\item type \textit{jugadas} = \(\TLista{\textit{pos} \times \ent}\) 
	\item type \textit{banderitas} = \(\TLista{\textit{pos}}\) 
\end{description}
\newpage


% FUNCIONES AUXILIARES%%%%%%%%%%%%%%%%%%%%%%%%%%
\section{Funciones Auxiliares Y Predicados}

\aux{max}{x,y: \ent}{\ent}{ if(x < y)\ then\ y\ else\ x}

\aux{min}{x,y: \ent}{\ent}{ if(x < y)\ then\ x\ else\ y}

\aux{es1SiPosicionEsBombaSino0}{t: \textit{tablero}, x, y: \ent}{\ent}{if(t[x][y]=true)\ then\ 1\ else\ 0\ fi;}

\aux{cantidadTotalDeBombas}{t: \textit{tablero}}{\ent}{\sum_{i = 0}^{\longitud{t}-1} \sum_{j = 0}^{\longitud{t}-1} if(t[i][j] = true\ then\ 1\ else\ 0\ fi;)}

\aux{cantidadDeBombasEnPosicionesDeLaJugada}{t: \textit{tablero}, j: \textit{jugada}}{\ent}{
	\newline \sum^{\longitud{j}-1}_{i=0}\ es1SiPosicionEsBombaSino0(t,\ j[i][0][0], j[1][0][1])
}


\aux{posicionesSinMinasTotales}{t: \textit{tablero}}{\ent}{
	(\sum_{i = 0}^{\longitud{t}-1} \sum_{k = 0}^{\longitud{t}-1} if(t[i][k]\ =\ false)\ then\ 1\ else\ 0\ fi;)
}

% Ejercicio 2
\bigbreak
\pred{tableroValido}{t: \textit{tablero}}{
	(\forall i: \ent)(0 \leq i < \longitud{t} \implicaLuego \longitud{t}\ =\ \longitud{t[i]})\ \wedge (cantidadTotalDeBombas(t)\ >\ 0)
}

\bigbreak
\pred{posicionValida}{t: \textit{tablero}, j: \textit{jugada}}{
	(\forall i: \ent)(0 \leq i < \longitud{j} \implicaLuego (0 \leq j[i]_{00} < \longitud{t}\ \wedge\ 0 \leq j[i]_{01} < \longitud{t} ))
}

\bigbreak
\pred{todasLasPosicionesDeLaJugadaSonValidas}{t: \textit{tablero}, j: \textit{jugada}}{
	(\forall i:\ent)(0\ \leq i\ <\ \longitud{j}\ \implicaLuego\ posicionValida(t, j))
}

\bigbreak
\pred{noExistenPosicionesRepetidasEnLaJugada}{j: \textit{jugadas}}{
	(\forall i: \ent)(\forall k: \ent)((0\ \leq\ i <\ \longitud{j}\ \wedge\ 0\ \leq\ k\ <\ \longitud{j}\ \wedge\ i \neq k) \newline
	\implicaLuego j[i][0]\ \neq\ j[k][0])
}

\bigbreak
\pred{esLaCantidadDeMinasAdyacentesCorrectaParaTodaLaJugada}{t: \textit{tablero}, j: \textit{jugadas}}{
	((\forall i: \ent)(0\ \leq\ i\ <\ \longitud{j}\ \implicaLuego\ j[i][1]\ =\ minasAdyacentes(t, j[i][0])))
}

% Ejercicio 3
\bigbreak
\pred{posicionPerteneceATablero}{p: \textit{pos}, t: \textit{tablero}}{
	(0\ \leq \ p_{0}\ < \longitud{t})\ \wedge\ ( 0\ \leq\ p_{1}\ < \longitud{t})
}

\bigbreak
\pred{posicionNoPerteneceAJugadas}{p: \textit{pos}, j: \textit{jugada}}{
	(\forall i: \ent)(0\ \leq\ i <  \<\ \longitud{j}\ \implicaLuego\ j[i][0] \neq p)
}

\bigbreak
\pred{posicionNoPerteneceABanderitas}{p: \textit{posicion}, b: \textit{banderita}}{\#apariciones(b,p)\ =\ 0}

\bigbreak
\pred{banderitasValidasParaLaJugada}{b: \textit{banderitas}, j: \textit{jugadas}, t: \textit{tablero}}{
	noExistenPosicionesRepetidasEnLasBanderitas(b)\ \wedge\ \newline
	todasLasPosicionesDeLasBanderitasSonValidas(t,\ b)\ \wedge \newline
	ningunaPosicionDeLaJugadaEstaEnBanderitas(j,\ b)
}

\bigbreak
\pred{noExistenPosicionesRepetidasEnLasBanderitas}{b: \textit{banderitas}}{
	(\forall i: \ent)(0\ \leq\ i\ <\ \longitud{b}\ \implicaLuego\ (\#apariciones(b,\ b[i])\ =\ 1))
}

\bigbreak
\pred{todasLasPosicionesDeLasBanderitasSonValidas}{t: \textit{tablero}, b: \textit{banderitas}}{
	(\forall i: \ent)(0\ \leq\ i\ <\ \longitud{b}\ \implicaLuego\ posicionPerteneceATablero(b[i],\ t))
}

\bigbreak
\pred{ningunaPosicionDeLaJugadaEstaEnBanderitas}{j: \textit{jugadas}, b: \textit{banderitas}}{
	(\forall i: \ent)(0\ \leq\ i\ <\ \longitud{j}\ \implicaLuego\ (\#apariciones(b,\ j[i][0]) =\ 0))
}

\bigbreak
\pred{posicionPerteneceABanderitas}{p: \textit{pos}, b: \textit{banderitas}}{
	(\exists i:\ent)(0\ \leq\ i < \longitud{b}\ \yLuego\ b[i]\ =\ p)
}


\bigbreak
\pred{\(todasLasPosicionesDeS_{1}PertenecenAS_{2}\)}{\(b_{1}\), \(b_{2}\): \textit{banderitas}}{
	(\forall x: pos)(posicionPerteneceABanderitas(x, b_{1})\ \implica\ posicionPerteneceABanderitas(x, b_{2}))
}

\bigbreak
\pred{jugadasNoRepetidas}{j: \textit{jugadas}}{
	(\forall i: \ent)(\forall k: \ent)((0 \leq i < \longitud{j}\ \wedge\ 0 \leq k < \longitud{j}\ \wedge\ i \neq j) \implicaLuego j[i]_{0} \neq j[k]_{0})
}
		
% Ejercicio 4

% Ejercicio 5
\bigbreak
\pred{jugadasTodasLasPosicionesSinBombas}{t: \textit{tablero}, j: \textit{jugadas}}{
	\longitud{j}\ =\ posicionesSinMinasTotales(t)
}



% Ejercicio 6
\bigbreak
\pred{noGan\'oNiPerdi\'o}{j: \textit{jugadas}, t: \textit{tablero}}{
	(juegoValido \yLuego\ (cantidadDeBombasEnPosicionesDeLaJugada(j, t)\ =\ 0)\ \wedge\ (\longitud{j}\ <\ posicionesSinMinasTotales(t)))
}

\bigbreak
\pred{\(todasLasJ_{1}PertenecenAJ_{2}\)}{\(j_{1}\), \(j_{2}\): \textit{jugadas}}{
	(\forall x: pos)(posicionPerteneceAJugadas(x, j_{1})\ \implica\ posicionPerteneceAJugadas(x, j_{2}))
}

\bigbreak
\pred{posicionPerteneceAJugadas}{x: \ent, j: \textit{jugadas}}{
	(\exists i:\ent)(0\ \leq\ i < \longitud{j}\ \yLuego\ j[i]\ =\ x)
}


% Ejercicio 7
\bigbreak
\pred{posicionPerteneceALaSecuencia}{p: \textit{pos}, s: \TLista{pos}}{
	(\exists i:\ent)(0\ \leq\ i < \longitud{s}\ \yLuego\ s[i]\ =\ p)
}

\bigbreak
\pred{secuenciaOrdenada}{\(p_{1}\), \(p_{2}\): \textit{pos}, s: \TLista{pos}}{
	(s[0]\ =\ p_{1}\ \wedge\ s[\longitud{s}\ -\ 1]\ =\ p_{2})\ \wedge \newline
	(\forall i: \ent)(0\ \leq\ i\ <\ \longitud{s} - 1\ \implicaLuego\ esAdyacente(p[i],\ p[i+1]))
}

\bigbreak
\pred{esPermutacion}{s, t: \TLista{pos}}{
	(\longitud{s}\ =\ \longitud{t})\ \wedge\ \newline 
	((\forall x: \textit{pos})(\#apariciones(x, s) = \#apariciones(x,t)))
}
% Ejercicio 8

% Ejercicio 9


\bigbreak
\pred{sonAdyacentes}{\(p_{1}\), \(p_{2}\): \textit{pos}}{
	(\longitud{p_{1}[0]-p_{2}[0]}\ =\ 1\ \wedge\ \longitud{p_{1}[1]-p_{2}[1]}\ =\ 0)\ \vee \newline
	(\longitud{p_{1}[0]-p_{2}[0]}\ =\ 0\ \wedge\ \longitud{p_{1}[1]-p_{2}[1]}\ =\ 1)
}

\bigbreak
\pred{hayPatron121}{t: \textit{tablero}, j: \textit{jugadas}}{
	(\exists s: \TLista{pos})(esPatron121(t,j,s)\ \yLuego\ \newline 
	(\exists p:pos)(posicionPerteneceATablero(p,t)\ \yLuego  
	posicionNoPerteneceAJugada(p,j)\ \wedge\ \newline 
	(\forall i: \ent)(0\ \leq i\ <\ \longitud{s}\ \implicaLuego\ sonAdyacentes(p, s[i])))
}

\bigbreak
\pred{esPatron121}{t: \textit{tablero}, j: \textit{jugadas}, s: \TLista{pos}}{
	posicionesPertenecenATablero(t,s)\ \wedge\ posicionesPertenecenAJugadas(j,s)\ \wedge\ (\longitud{s}\ =\ 3)\ \yLuego\ \newline 
	(\exists\ t: \TLista{pos})(secuenciaOrdenada(t)\ \wedge\ esPermutacion(s,t)\ \yLuego\ cumple121(s));
}

\bigbreak
\pred{posicionesPertenecenATablero}{t: \textit{tablero}, s: \TLista{pos}}{
	(\forall i: \ent)(0\ \leq i\ < \longitud{s}\ \implicaLuego\ posicionPerteneceATablero(s[i],t))
}

\bigbreak
\pred{posicionesPertenecenAJugadas}{j: \textit{jugadas}, s: \TLista{pos}}{
	(\forall i: \ent)(0\ \leq i\ < \longitud{s}\ \implicaLuego\ posicionPerteneceAJugadas(s[i], j)
}

\bigbreak
\pred{cumple121}{s: \TLista{pos}}{
	((minasAdyacentes(s[0]) = 1)\ \wedge\ (minasAdyacentes(s[1]) = 2)\ \wedge\ (minasAdyacentes(s[2]) = 1))
}
















%%%%%%%%%%%%%%%%%%%%%%%%%%%%%%%%%%%%%%%%%%%%%%%%
\newpage




% PROBLEMAS % 
\section{Problemas}

% PARTE I-----------------------------------------------%
\subsection{Parte I: Juego b\'asico}


% -------------------------------------------EJERCICIO 1-------------------------------------------%
\subsubsection{Ejercicio 1}
\begin{addmargin}[4em]{1em}
\aux{minasAdyacentes}{t: \textit{tablero}, p: \textit{pos}}{\ent}{
	\newline
	\sum_{i = max(p[0] - 1,\ 0)}^{min(p[0] + 1,\ \longitud{t} - 1)} \sum_{j = max(p[1] - 1,\ 0)}^{min(p[1] + 1,\ \longitud{t} - 1)}
	\ (es1SiPosicionEsBombaSino0(t,\ i,\ j)\ -
	\newline es1SiPosicionEsBombaSino0(t,\ p[0],\ p[1]))}
\end{addmargin}





% -------------------------------------------EJERCICIO 2-------------------------------------------%
\subsubsection{Ejercicio 2}
\begin{addmargin}[4em]{1em}
\pred{juegoValido}{t: \textit{tablero}, j: \textit{jugadas}}{
	(tableroValido(t)\ \wedge 
	\\ todasLasPosicionesDeLaJugadaSonValidas(t, j)\ \wedge
	\\ noExistenPosicionesRepetidasEnLaJugada(j))\ \yLuego
	\\ (esLaCantidadDeMinasAdyacentesCorrectaParaTodaLaJugada(t,\ j)\ \wedge\ 
	\newline cantidadDeBombasEnPosicionesDeLaJugada(t,\ j)\ \leq 1)
}
\end{addmargin}




% -------------------------------------------EJERCICIO 3-------------------------------------------%
\subsubsection{Ejercicio 3}
\begin{addmargin}[4em]{1em}
\begin{proc}{plantarBanderita}{in t: \textit{tablero} , in j: \textit{jugadas}, in p: \textit{pos}, inout b: \textit{banderitas}}{}
	\pre{juegoValido(t,\ j) \wedge\ \newline 
	posicionPerteneceATablero(p, t)\ \wedge \newline 
	posicionNoPerteneceAJugada(p, j)\ \wedge \newline 
	posicionNoPerteneceABanderitas(p, b)\ \wedge \newline 
	banderitaValidaParaLaJugada(b,\ j,\ t)\ \wedge \newline 
	b = b_{0}}
	\post{posicionPerteneceABanderitas(p,\ b)\ \wedge \newline
	todassLasPosicionesDeS_{1}PertenecenAS_{2}(b_{0},\ b)\ \wedge \newline 
	(\longitud{b} = \longitud{b_{0}}\ +\ 1)}
\end{proc}
\end{addmargin}




% -------------------------------------------EJERCICIO 4-------------------------------------------%
\subsubsection{Ejercicio 4}
\begin{addmargin}[4em]{1em}
\begin{proc}{perdi\'o}{in t: \textit{tablero} , in j: \textit{jugadas}, out res: $\bool$}{}
	\pre{juegoValido(t,j)}
	\post{res = true \iff\ cantidadDeBombasEnPosicionesDeLaJugada(t,\ j) = 1}

\end{proc}
\end{addmargin}


% -------------------------------------------EJERCICIO 5-------------------------------------------%
\subsubsection{Ejercicio 5}
\begin{addmargin}[4em]{1em}
\begin{proc}{gan\'o}{in t: \textit{tablero} , in j: \textit{jugadas}, out res: $\bool$}{}
	\pre{juegoValido(t,j)}
	\post{res = true \iff\ cantidadDeBombasEnPosicionesDeLaJugada(t,\ j) = 0\ \wedge \newline
	jugadasTodasLasPosicionesSinBombas(t,\ j)}
\end{proc}
\end{addmargin}

\bigbreak
% -------------------------------------------EJERCICIO 6-------------------------------------------%
\subsubsection{Ejercicio 6}
\begin{addmargin}[4em]{1em}
\begin{proc}{jugar}{in t: \textit{tablero} , in b: \textit{banderitas}, in p: \textit{pos}, inout j: \textit{jugadas}}{}
	\pre{noGanoNiPerdio(j,\ t)\ \wedge\ \newline
	posicionNoPerteneceAJugada(p,\ j)\ \wedge\ \newline
	posicionNoPerteneceABanderitas(p,\ b)\ \wedge\ \newline
	posicionPerteneceATablero(p,\ t)\ \wedge\ \newline
	banderitasValidasParaLaJugada(b,\ j,\ t)}
	\post{\neg posicionNoPerteneceAJugada(p, j)\ \wedge\ \newline
	todasLasJ_{1}PertenecenAJ_{2}(j_{0},\ j)\ \wedge\ \newline
	(\longitud{j}\ = \longitud{j_{0}}\ +\ 1)}
\end{proc}
\end{addmargin}


% PARTE II----------------------------------------------%
\newpage
\subsection{Parte II: Despejar los vac\'ios}
% -------------------------------------------EJERCICIO 7-------------------------------------------%
\subsubsection{Ejercicio 7}
\begin{addmargin}[4em]{1em}
\pred{caminoLibre}{t: \textit{tablero}, \(p_{0}\): \textit{pos}, \(p_{1}\): \textit{pos}}{
	(\exists s:\ seq<pos>)((posicionPerteneceALaSecuencia(p_{0},s)\ \wedge\ \newline
	posicionPerteneceALaSecuencia(p_{1}, s)\ \wedge\ \newline
	(\forall p: pos)(posicionPerteneceALaSecuencia(p,s) \implica posicionPerteneceATablero(p,t))\ \yLuego\ \newline
	(\forall p: pos)(posicionPerteneceALaSecuencia(p,s)\ \wedge\ p \neq p_{1} \implica minasAdyacentes(t,p) = 0)\ \wedge\ \newline 
	((8 > minasAdyacentes(t,p_{1}) \leq 1)\ \wedge \newline
	(\exists s_{2}:\ seq<pos>)(secuenciaOrdenada(p_{0}, p_{1}, s_{2})\ \wedge\ esPermutacion(s, s_{2})))
}
\end{addmargin}

% -------------------------------------------EJERCICIO 8-------------------------------------------%
\subsubsection{Ejercicio 8}
\begin{addmargin}[4em]{1em}
\begin{proc}{jugarPlus}{in t: \textit{tablero} , in b: \textit{banderitas},
	                    in p: \textit{pos},
	                    inout j: \textit{jugadas}}{}{ hola

}

\end{proc}
\end{addmargin}


% PARTE III---------------------------------------------%
\newpage
\subsection{Parte III: Jugador autom\'atico}
% -------------------------------------------EJERCICIO 9-------------------------------------------%
\subsubsection{Ejercicio 9}
\begin{addmargin}[4em]{1em}
\begin{proc}{sugerirAutom\'atico121}{in t: \textit{tablero} , in b: \textit{banderitas},
	                                 out p: \textit{pos}}{}{
	\pre{juegoValido(t,\ j)\ \wedge\ noGanoNiPerdio(j,\ t)\ \wedge \newline
	hayPatron121(t,\ j)}
	\post{posicionPerteneceATablero(p, t)\ \yLuego \newline 
	posicionNoPerteneceAJugadas(p, j) \wedge \newline
	(\exists\ s: \TLista{pos})(esPatron121(t, j, s)\ \yLuego\ sonAdyacentes(p, s[1]))
	}
}

\end{proc}
\end{addmargin}
\end{document}
