% CONFIGURACIÓN DEL PROYECTO
\documentclass[a4paper]{article}
\setlength{\parskip}{2mm}
\newcommand{\tab}{~ \qquad}
\input{Algo1Macros}
\usepackage{caratula} 
\usepackage{hyperref}
\usepackage{scrextend}
\usepackage{blindtext}




% CARÁTULA
\begin{document}

\titulo{TP de Especificación}
\subtitulo{Trabajo Práctico Grupal}
\fecha{30 de Marzo de 2022}
\materia{Algoritmos y Estructuras de Datos I}
\grupo{Grupo 10}

\newcommand{\dato}{\textit{Dato}}
\newcommand{\individuo}{\textit{Individuo}}

\integrante{Dominguez, Emilia}{37752993}{maemiliadominguez@gmail.com}
\integrante{Kerbs, Octavio}{64/22}{octaviokerbs@gmail.com}
\integrante{Russo, Gabriel}{107/19}{gabrielrussoguiot@gmail.com}
\integrante{Traverso, Lucas}{479/18}{lucas6246@gmail.com}

\maketitle


% ÍNDICE
\tableofcontents
\newpage


% DEFINICIÓN DE TIPOS
\section{Definición de Tipos}
\begin{description}
	\item type \textit{pos} = \(\ent \times \ent\) 
	\item type \textit{tablero} = \( \TLista{\TLista{\bool}}\) 
	\item type \textit{jugadas} = \(\TLista{\textit{pos} \times \ent}\) 
	\item type \textit{banderitas} = \(\TLista{\textit{pos}}\) 
\end{description}
\newpage


% FUNCIONES AUXILIARES--------------------------------------------------


% PROBLEMAS % 
\section{Problemas}

% PARTE I-----------------------------------------------%
\subsection{Parte I: Juego b\'asico}


% -------------------------------------------EJERCICIO 1-------------------------------------------%
\subsubsection{Ejercicio 1}
\begin{addmargin}[4em]{1em}
\aux{minasAdyacentes}{t: \textit{tablero}, p: \textit{pos}}{\ent}{
	\newline
	\sum_{i = max(p[0] - 1,\ 0)}^{min(p[0] + 1,\ \longitud{t} - 1)} \sum_{j = max(p[1] - 1,\ 0)}^{min(p[1] + 1,\ \longitud{t} - 1)}
	\ es1SiPosicionEsBombaSino0(t,\ i,\ j)\ -
	\newline es1SiPosicionEsBombaSino0(t,\ p[0],\ p[1])}
\end{addmargin}





% -------------------------------------------EJERCICIO 2-------------------------------------------%
\subsubsection{Ejercicio 2}
\begin{addmargin}[4em]{1em}
\pred{juegoValido}{t: \textit{tablero}, j: \textit{jugadas}}{
	(tableroValido(t)\ \wedge \newline
	todasLasPosicionesDeLaJugadaPertenecenAlTablero(t, j)\ \wedge \newline
	noExistenPosicionesRepetidasEnLaJugada(j))\ \yLuego \newline
	(esLaCantidadDeMinasAdyacentesCorrectaParaTodaLaJugada(t,\ j)\ \wedge\ 
	\newline cantidadDeBombasEnPosicionesDeLaJugada(t,\ j)\ \leq 1)
}
\end{addmargin}




% -------------------------------------------EJERCICIO 3-------------------------------------------%
\subsubsection{Ejercicio 3}
\begin{addmargin}[4em]{1em}
\begin{proc}{plantarBanderita}{in t: \textit{tablero} , in j: \textit{jugadas}, in p: \textit{pos}, inout b: \textit{banderitas}}{}{
	\pre{juegoValido(t,\ j) \wedge\ \newline 
	posicionPerteneceAlTablero(t,\ p)\ \wedge \newline 
	\neg\ posicionPerteneceAJugadas(p,\ j)\ \wedge \newline 
	\neg\ posicionPerteneceABanderitas(p,\ b)\ \wedge \newline 
	banderitasValidasParaLaJugada(b,\ j,\ t)\ \wedge \newline 
	b = b_{0}}
	\post{posicionPerteneceABanderitas(p,\ b)\ \wedge \newline
	todasLasPosicionesDeB_{1}PertenecenAB_{2}(b_{0},\ b)\ \wedge \newline 
	(\longitud{b} = \longitud{b_{0}}\ +\ 1)}
}
\end{proc}
\end{addmargin}




% -------------------------------------------EJERCICIO 4-------------------------------------------%
\subsubsection{Ejercicio 4}
\begin{addmargin}[4em]{1em}
\begin{proc}{perdi\'o}{in t: \textit{tablero} , in j: \textit{jugadas}, out res: $\bool$}{}{
	\pre{juegoValido(t,\ j)}
	\post{res = true \iff\ cantidadDeBombasEnPosicionesDeLaJugada(t,\ j) = 1}
}
\end{proc}
\end{addmargin}


% -------------------------------------------EJERCICIO 5-------------------------------------------%
\subsubsection{Ejercicio 5}
\begin{addmargin}[4em]{1em}
\begin{proc}{gan\'o}{in t: \textit{tablero} , in j: \textit{jugadas}, out res: $\bool$}{}
	\pre{juegoValido(t,\ j)}
	\post{res = true \iff\ cantidadDeBombasEnPosicionesDeLaJugada(t,\ j) = 0\ \wedge \newline
	jugadasTodasLasPosicionesSinBombas(t,\ j)}
\end{proc}
\end{addmargin}

\bigbreak
% -------------------------------------------EJERCICIO 6-------------------------------------------%
\subsubsection{Ejercicio 6}
\begin{addmargin}[4em]{1em}
\begin{proc}{jugar}{in t: \textit{tablero} , in b: \textit{banderitas}, in p: \textit{pos}, inout j: \textit{jugadas}}{}{
	\pre{juegoValido(t,\ j) \wedge\ \newline 
	posicionPerteneceAlTablero(t,\ p)\ \wedge \newline 
	\neg\ posicionPerteneceAJugadas(p,\ j)\ \wedge \newline 
	\neg\ posicionPerteneceABanderitas(p,\ b)\ \wedge \newline 
	banderitasValidasParaLaJugada(b,\ j,\ t)\ \yLuego \newline 
	(juegoEnMarcha(j,\ t)\ \wedge\ \newline
	j = j_{0})}
	\post{posicionPerteneceAJugadas(p,\ j)\ \wedge\ \newline
	todasLasJ_{1}PertenecenAJ_{2}(j_{0},\ j)\ \wedge\ \newline
	(\longitud{j}\ = \longitud{j_{0}}\ +\ 1)}
}
\end{proc}
\end{addmargin}


% PARTE II----------------------------------------------%
\newpage
\subsection{Parte II: Despejar los vac\'ios}
% -------------------------------------------EJERCICIO 7-------------------------------------------%
\subsubsection{Ejercicio 7}
\begin{addmargin}[4em]{1em}
\pred{caminoLibre}{t: \textit{tablero}, \(p_{0}\): \textit{pos}, \(p_{1}\): \textit{pos}}{
	(\exists s:\ seq<pos>)(caminoLibreSinDefinirMinasAdyacentesALaUltimaPosicion(t,\ p_{0},\ p_{1},\ s)\ \yLuego\ \newline
	8 > minasAdyacentes(t,\ p_{1}) \geq 1)
}
\end{addmargin}

% -------------------------------------------EJERCICIO 8-------------------------------------------%
\subsubsection{Ejercicio 8}
\begin{addmargin}[4em]{1em}
\begin{proc}{jugarPlus}{in t: \textit{tablero} , in b: \textit{banderitas},
	                    in p: \textit{pos},
	                    inout j: \textit{jugadas}}{}{ 
	\pre{juegoValido(t,\ j) \wedge\ \newline 
	posicionPerteneceAlTablero(t,\ p)\ \wedge \newline 
	\neg\ posicionPerteneceAJugadas(p,\ j)\ \wedge \newline 
	\neg\ posicionPerteneceABanderitas(p,\ b)\ \wedge \newline 
	banderitasValidasParaLaJugada(b,\ j,\ t)\ \yLuego \newline 
	(juegoEnMarcha(j,\ t)\ \wedge\ \newline
	jugadasExtendidasValidas (t,\ j,\ b)\ \wedge\ \newline
	j = j_{0})}
	\post{(posicionPerteneceAJugadas(p,\ j)\ \wedge \newline
	todasLasJ_{1}PertenecenAJ_{2}(j_{0},\ j)\ \wedge\ \newline
	juegoValido(t,\ j)) \yLuego\ \newline
	(jugadasExtendidasValidas (t,\ j,\ b)\ \wedge\ \newline
	(\forall q: pos)(posicionPerteneceAlTablero(t,\ q)\ \wedge \newline 
	\neg\ posicionPerteneceAJugadas(q,\ j_{0})\ \wedge \newline 
	q \neq\ p \wedge \newline 
	\neg\ (\exists s: \TLista{pos})(caminoLibreSinDefinirMinasAdyacentesALaUltimaPosicionConBanderitas(t,\ p,\ q,\ s,\ b)) \implica\
	\neg\ posicionPerteneceAJugadas(q,\ j)))}
}

\end{proc}
\end{addmargin}


% PARTE III---------------------------------------------%
\newpage
\subsection{Parte III: Jugador autom\'atico}
% -------------------------------------------EJERCICIO 9-------------------------------------------%
\subsubsection{Ejercicio 9}
\begin{addmargin}[4em]{1em}
\begin{proc}{sugerirAutom\'atico121}{in t: \textit{tablero} , in b: \textit{banderitas},
	                                 out p: \textit{pos}}{}{ 

	\pre{juegoValido(t,\ j)\ \wedge\ juegoEnMarcha(j,\ t)\ \wedge \newline
	hayPatron121(t,\ j)}
	\post{posicionPerteneceAlTablero(p, t)\ \yLuego \newline 
	posicionNoPerteneceAJugadas(p, j) \wedge \newline
	(\exists\ s: \TLista{pos})(esPatron121(t, j, s)\ \yLuego\ sonAdyacentesNoDiagonales(p, s[1]))
	}
}

\end{proc}
\end{addmargin}
\newpage



\section{Funciones Auxiliares Y Predicados}



\subsection{Ejercicio 1}
\aux{es1SiPosicionEsBombaSino0}{t: \textit{tablero}, x, y: \ent}{\ent}{if(t[x][y]=true)\ then\ 1\ else\ 0\ fi;}
\aux{max}{x,y: \ent}{\ent}{ if(x < y)\ then\ y\ else\ x}
\aux{min}{x,y: \ent}{\ent}{ if(x < y)\ then\ x\ else\ y}



\subsection{Ejercicio 2}
\bigbreak
\pred{tableroValido}{t: \textit{tablero}}{
	(\forall i: \ent)(0\ \leq i\ <\ \longitud{t} \implicaLuego \longitud{t}\ =\ \longitud{t[i]})\ \yLuego\ (cantidadTotalDeBombas(t)\ >\ 0)
}
\bigbreak
\aux{cantidadTotalDeBombas}{t: \textit{tablero}}{\ent}{\sum_{i = 0}^{\longitud{t}-1} \sum_{j = 0}^{\longitud{t}-1}\ es1SiPosicionEsBombaSino0(\ t,\ i,\ j)}
\bigbreak
\pred{todasLasPosicionesDeLaJugadaPertenecenAlTablero}{t: \textit{tablero}, j: \textit{jugada}}{
	(\forall i:\ent)(0\ \leq\ i\ <\ \longitud{j}\ \implicaLuego\ posicionPerteneceAlTablero(t, j[i][0]))
}
\bigbreak
\pred{posicionPerteneceAlTablero}{p: \textit{pos}, t: \textit{tablero}}{
	(0\ \leq \ p_{0}\ < \longitud{t})\ \wedge\ ( 0\ \leq\ p_{1}\ < \longitud{t})
}
\bigbreak
\pred{noExistenPosicionesRepetidasEnLaJugada}{j: \textit{jugadas}}{
	(\forall i: \ent)(\forall k: \ent)((0\ \leq\ i <\ \longitud{j}\ \wedge\ 0\ \leq\ k\ <\ \longitud{j}\ \wedge\ i \neq k) \newline
	\implicaLuego\ j[i][0]\ \neq\ j[k][0])
}
\bigbreak
\pred{esLaCantidadDeMinasAdyacentesCorrectaParaTodaLaJugada}{t: \textit{tablero}, j: \textit{jugadas}}{
	(\forall i: \ent)(0\ \leq\ i\ <\ \longitud{j}\ \implicaLuego\ j[i][1]\ =\ minasAdyacentes(t, j[i][0]))
}



\subsection{Ejercicio 3}

\bigbreak
\pred{posicionPerteneceAJugadas}{p: \textit{pos}, j: \textit{jugada}}{
	(\exists i: \ent)(0\ \leq\ i\ <\  \longitud{j}\ \yLuego\ j[i][0]\ =\ p)
}
\bigbreak
\pred{posicionPerteneceABanderitas}{p: \textit{posicion}, b: \textit{banderita}}{
	(\exists i: \ent)(0\ \leq\ i\ <\  \longitud{b}\ \yLuego\ b[i]\ =\ p)
}
\bigbreak
\pred{banderitasValidasParaLaJugada}{b: \textit{banderitas}, j: \textit{jugadas}, t: \textit{tablero}}{
	noExistenPosicionesRepetidasEnLasBanderitas(b)\ \wedge\ \newline
	todasLasPosicionesDeLasBanderitasPertenecenAlTablero(t,\ b)\ \wedge \newline
	ningunaPosicionDeLaJugadaEstaEnBanderitas(j,\ b)
}
\bigbreak
\pred{noExistenPosicionesRepetidasEnLasBanderitas}{b: \textit{banderitas}}{
	(\forall i: \ent)(\forall k: \ent)((0\ \leq\ i <\ \longitud{b}\ \wedge\ 0\ \leq\ k\ <\ \longitud{b}\ \wedge\ i \neq k) \newline
	\implicaLuego\ b[i]\ \neq\ b[k])
}
\bigbreak
\pred{todasLasPosicionesDeLasBanderitasPertenecenAlTablero}{t: \textit{tablero}, b: \textit{banderitas}}{
	(\forall i: \ent)(0\ \leq\ i\ <\ \longitud{b}\ \implicaLuego\ posicionPerteneceAlTablero(t,\ b[i]))
}
\bigbreak
\pred{ningunaPosicionDeLaJugadaEstaEnBanderitas}{j: \textit{jugadas}, b: \textit{banderitas}}{
	(\forall i: \ent)(\forall k: \ent)((0\ \leq\ i\ <\ \longitud{j}\ \wedge\ 0\ \leq\ k\ <\ \longitud{b})\ \implicaLuego\ j[i][0]\ \neq\ b[k])
}
\bigbreak
\pred{\(todasLasPosicionesDeB_{1}PertenecenAB_{2}\)}{\(b_{1}\), \(b_{2}\): \textit{banderitas}}{
	(\forall x: pos)(posicionPerteneceABanderitas(x, b_{1})\ \implica\ posicionPerteneceABanderitas(x, b_{2}))
}



\subsection{Ejercicio 4}
\aux{cantidadDeBombasEnPosicionesDeLaJugada}{t: \textit{tablero}, j: \textit{jugada}}{\ent}{
	\newline \sum^{\longitud{j}-1}_{i=0}\ es1SiPosicionEsBombaSino0(\ t,\ j[i][0][0],\ j[i][0][1])}



\subsection{Ejercicio 5}
\bigbreak
\pred{jugadasTodasLasPosicionesSinBombas}{t: \textit{tablero}, j: \textit{jugadas}}{
	\longitud{j}\ =\ posicionesSinMinas(t)
}
\aux{posicionesSinMinas}{t: \textit{tablero}}{\ent}{
	(\sum_{i = 0}^{\longitud{t}-1} \sum_{k = 0}^{\longitud{t}-1} if(t[i][k]\ =\ false)\ then\ 1\ else\ 0\ fi;)}



\subsection{Ejercicio 6}
\bigbreak
\pred{juegoEnMarcha}{j: \textit{jugadas}, t: \textit{tablero}}{
	cantidadDeBombasEnPosicionesDeLaJugada(t,\ j)\ =\ 0\ \wedge\ \neg\ jugadasTodasLasPosicionesSinBombas(t,\ j)
}
\bigbreak
\pred{\(todasLasJ_{1}PertenecenAJ_{2}\)}{\(j_{1}\), \(j_{2}\): \textit{jugadas}}{
	(\forall x: pos)(posicionPerteneceAJugadas(x, j_{1})\ \implica\ posicionPerteneceAJugadas(x, j_{2}))
}



\subsection{Ejercicio 7}
\bigbreak
\pred{caminoLibreSinDefinirMinasAdyacentesALaUltimaPosicionConBanderitas}{t: \textit{tablero}, \(p_{0}\): \textit{pos}, \(p_{1}\): \textit{pos},\ s: \TLista{pos},\ b: \textit{banderitas}}{
	(posicionPerteneceASecuencia(p_{0},\ s)\ \wedge\ \newline
	posicionPerteneceASecuencia(p_{1},\ s)\ \wedge\ \newline
	(\forall p: pos)(posicionPerteneceASecuencia(p,\ s) \implica posicionPerteneceAlTablero(t,\ p)))\ \yLuego\ \newline
	((\forall p: pos)(posicionPerteneceASecuencia(p,\ s)\ \wedge\ p \neq p_{1} \implica minasAdyacentes(t,\ p) = 0)\ \wedge\ \newline 
	(\exists s_{2}:\ seq<pos>)(ningunaPosicionDeLaSecuenciaEstaEnBanderitas(s_{2},\ b)\ \wedge\ \newline 
	secuenciaDePosicionesAdyacentes(p_{0}, p_{1}, s_{2})\ \wedge\ esPermutacion(s, s_{2})))
}
\bigbreak
\pred{ningunaPosicionDeLaSecuenciaEstaEnBanderitas}{s: \TLista{pos},\ b: \textit{banderitas}}{
	(\forall i: \ent)(\forall k: \ent)((0\ \leq\ i\ <\ \longitud{b}\ \wedge\ 0\ \leq\ k\ <\ \longitud{s})\ \implicaLuego\ s[k]\ \neq\ b[i])
}
\bigbreak
\pred{posicionPerteneceASecuencia}{p: \textit{pos}, s: \TLista{pos}}{
	(\exists i:\ent)(0\ \leq\ i\ < \longitud{s}\ \yLuego\ s[i]\ =\ p)
}
\bigbreak
\pred{secuenciaDePosicionesAdyacentes}{\(p_{1}\), \(p_{2}\): \textit{pos}, s: \TLista{pos}}{
	(s[0]\ =\ p[1]\ \wedge\ s[\longitud{s}\ -\ 1]\ =\ p[2])\ \wedge \newline
	(\forall i: \ent)(0\ \leq\ i\ <\ \longitud{s}\ -\ 1\ \implicaLuego\ esAdyacente(p[i],\ p[i+1]))
}



\subsection{Ejercicio 8}
\bigbreak
\pred{jugadasExtendidasValidas}{t: \textit{tablero}, j: \textit{jugadas}, b: \textit{banderitas}}{
	(\forall p: \textit{pos})(posicionPerteneceAJugadas(p,\ j)\ \implicaLuego\ \newline 
	((0\ =\ minasAdyacentes(t,\ p) \wedge\ \newline
	todasLasPosicionesConMinasAdyacentesYCaminoLibreAPertenecenAJugada(p,\ j,\ b)) \lor \newline 
	(0\ \neq\ minasAdyacentes(t,\ p) \wedge\ \newline
	todasLasPosicionesSinMinasAdyacentesConUnaPosicionConfirmadaEnJugadaYCaminoLibreAPertenecenAJugada(p,\ j,\ b))))
}
\bigbreak
\pred{todasLasPosicionesConMinasAdyacentesYCaminoLibreAPertenecenAJugada}{t: \textit{tablero},\ p: \textit{posicion},\ j: \textit{jugada},\ b: \textit{banderitas}}{
	(\forall q: pos)(((posicionPerteneceAlTablero(t,\ q) \wedge \newline
	\neg\ posicionPerteneceABanderitas(q,\ b))\ \yLuego \newline 
	(0\ \neq\ minasAdyacentes(t,\ q) \wedge\ \newline
	caminoLibreConBanderitas(t,\ p,\ q,\ b))) \implicaLuego \newline
	posicionPerteneceAJugadas (q,\ j))
}
\bigbreak
\pred{esAdyacente}{p,\ q: \textit{pos}}{
	q[0]\ -\ 1\ \leq\ p[0]\ \leq\ q[0]\ +\ 1 \wedge\ q[1]\ -\ 1\ \leq\ p[1]\ \leq\ q[1]\ +\ 1 \wedge\ p\ \neq\ q
}
\bigbreak
\pred{caminoLibreConBanderitas}{t: \textit{tablero}, \(p_{0}\): \textit{pos}, \(p_{1}\): \textit{pos},\ b: \textit{banderitas}}{
	(\exists s:\ seq<pos>)(caminoLibreSinDefinirMinasAdyacentesALaUltimaPosicionconBanderitas(t,\ p_{0},\ p_{1},\ s,\ b)\ \yLuego\ \newline
	8 > minasAdyacentes(t,\ p_{1}) \geq 1)
}
\bigbreak
\pred{todasLasPosicionesSinMinasAdyacentesConUnaPosicionConfirmadaEnJugadaYCaminoLibreAPertenecenAJugada}{t: \textit{tablero},\ p: \textit{posicion},\ j: \textit{jugada},\ b: \textit{banderitas}}{
	(\forall q: pos)(((posicionPerteneceAlTablero(t,\ q) \wedge \newline
	\neg\ posicionPerteneceABanderitas(q,\ b))\ \yLuego \newline 
	(0\ =\ minasAdyacentes(t,\ q) \wedge\ \newline
	caminoLibreConBanderitas(t,\ p,\ q,\ b) \wedge\ \newline
	existeUnaPosicionDelCaminoLibreQuePerteneceAJugada(t,\ p,\ q,\ j,\ b))) \implicaLuego \newline
	posicionPerteneceAJugadas(q,\ j))
}
\bigbreak
\pred{existeUnaPosicionDelCaminoLibreQuePerteneceAJugada}{t: \textit{tablero},\ p: \textit{posicion}, q: \textit{posicion},\ j: \textit{jugadas},\ b: \textit{banderitas}}{
	(\exists m: pos)((posicionPerteneceAlTablero(t,\ m) \wedge \newline
	posicionPerteneceAJugadas(m,\ j))\ \yLuego \newline 
	(0\ =\ minasAdyacentes(t,\ m) \wedge\ \newline
	caminoLibreConBanderitas(t,\ m,\ p,\ b) \wedge\ \newline
	(\exists s: \TLista{pos})(caminoLibreSinDefinirMinasAdyacentesALaUltimaPosicionConBanderitas(t,\ q,\ m,\ s,\ b))))
}





\subsection{Ejercicio 9}
\bigbreak
\pred{hayPatron121}{t: \textit{tablero}, j: \textit{jugadas}}{
	(\exists s: \TLista{pos})(esPatron121(t,j,s)\ \yLuego\ \newline 
	(\exists p:pos)(posicionPerteneceAlTablero(p,t)\ \yLuego  
	posicionNoPerteneceAJugadas(p,j)\ \wedge\ \newline 
	(\forall i: \ent)(0\ \leq i\ <\ \longitud{s}\ \implicaLuego\ sonAdyacentesNoDiagonales(p, s[i])))
}
\pred{esPatron121}{t: \textit{tablero}, j: \textit{jugadas}, s: \TLista{pos}}{
	posicionesPertenecenATablero(t,s)\ \wedge\ posicionesPertenecenAJugadas(j,s)\ \wedge\ (\longitud{s}\ =\ 3)\ \yLuego\ \newline 
	(\exists\ t: \TLista{pos})(secuenciaOrdenada(t)\ \wedge\ esPermutacion(s,t)\ \yLuego\ cumple121(s));
}
\bigbreak
\pred{posicionNoPerteneceAJugadas}{p: \textit{pos}, j: \textit{jugada}}{
	(\forall i: \ent)(0\ \leq\ i <  \<\ \longitud{j}\ \implicaLuego\ j[i][0] \neq p)
}
\bigbreak 
\pred{sonAdyacentesNoDiagonales}{\(p_{1}\), \(p_{2}\): \textit{pos}}{
	((p_{1}[0]-p_{2}[0])\ =\ 1\ \wedge\ (p_{1}[1]-p_{2}[1])\ =\ 0)\ \vee \newline
	((p_{1}[0]-p_{2}[0])\ =\ 0\ \wedge\ (p_{1}[1]-p_{2}[1])\ =\ 1)\ \vee \newline
	((p_{1}[0]-p_{2}[0])\ =\ -1\ \wedge\ (p_{1}[1]-p_{2}[1])\ =\ 0)\ \vee \newline
	((p_{1}[0]-p_{2}[0])\ =\ 0\ \wedge\ (p_{1}[1]-p_{2}[1])\ =\ -1)
}
\bigbreak
\pred{posicionesPertenecenATablero}{t: \textit{tablero}, s: \TLista{pos}}{
	(\forall i: \ent)(0\ \leq i\ < \longitud{s}\ \implicaLuego\ posicionPerteneceAlTablero(s[i],t))
}
\bigbreak
\pred{posicionesPertenecenAJugadas}{j: \textit{jugadas}, s: \TLista{pos}}{
	(\forall i: \ent)(0\ \leq i\ < \longitud{s}\ \implicaLuego\ posicionPerteneceAJugadas(s[i], j)
}
\bigbreak
\pred{cumple121}{s: \TLista{pos}}{
	((minasAdyacentes(s[0]) = 1)\ \wedge\ (minasAdyacentes(s[1]) = 2)\ \wedge\ (minasAdyacentes(s[2]) = 1))
}
\bigbreak
\pred{secuenciaOrdenada}{\(p_{1}\), \(p_{2}\): \textit{pos}, s: \TLista{pos}}{
	(s[0]\ =\ p_{1}\ \wedge\ s[\longitud{s}\ -\ 1]\ =\ p_{2})\ \wedge \newline
	(\forall i: \ent)(0\ \leq\ i\ <\ \longitud{s} - 1\ \implicaLuego\ esAdyacente(p[i],\ p[i+1]))
}
\bigbreak
\pred{esPermutacion}{s, t: \TLista{pos}}{
	(\longitud{s}\ =\ \longitud{t})\ \wedge\ \newline 
	(\forall x: \textit{pos})(posicionPerteneceASecuencia(p,\ s)\ \iff\ posicionPerteneceASecuencia(p,\ t))
}

\end{document}
